
% Default to the notebook output style

    


% Inherit from the specified cell style.




    
\documentclass{article}

    
    
    \usepackage{graphicx} % Used to insert images
    \usepackage{adjustbox} % Used to constrain images to a maximum size 
    \usepackage{color} % Allow colors to be defined
    \usepackage{enumerate} % Needed for markdown enumerations to work
    \usepackage{geometry} % Used to adjust the document margins
    \usepackage{amsmath} % Equations
    \usepackage{amssymb} % Equations
    \usepackage{eurosym} % defines \euro
    \usepackage[mathletters]{ucs} % Extended unicode (utf-8) support
    \usepackage[utf8x]{inputenc} % Allow utf-8 characters in the tex document
    \usepackage{fancyvrb} % verbatim replacement that allows latex
    \usepackage{grffile} % extends the file name processing of package graphics 
                         % to support a larger range 
    % The hyperref package gives us a pdf with properly built
    % internal navigation ('pdf bookmarks' for the table of contents,
    % internal cross-reference links, web links for URLs, etc.)
    \usepackage{hyperref}
    \usepackage{longtable} % longtable support required by pandoc >1.10
    \usepackage{booktabs}  % table support for pandoc > 1.12.2
    

    
    
    \definecolor{orange}{cmyk}{0,0.4,0.8,0.2}
    \definecolor{darkorange}{rgb}{.71,0.21,0.01}
    \definecolor{darkgreen}{rgb}{.12,.54,.11}
    \definecolor{myteal}{rgb}{.26, .44, .56}
    \definecolor{gray}{gray}{0.45}
    \definecolor{lightgray}{gray}{.95}
    \definecolor{mediumgray}{gray}{.8}
    \definecolor{inputbackground}{rgb}{.95, .95, .85}
    \definecolor{outputbackground}{rgb}{.95, .95, .95}
    \definecolor{traceback}{rgb}{1, .95, .95}
    % ansi colors
    \definecolor{red}{rgb}{.6,0,0}
    \definecolor{green}{rgb}{0,.65,0}
    \definecolor{brown}{rgb}{0.6,0.6,0}
    \definecolor{blue}{rgb}{0,.145,.698}
    \definecolor{purple}{rgb}{.698,.145,.698}
    \definecolor{cyan}{rgb}{0,.698,.698}
    \definecolor{lightgray}{gray}{0.5}
    
    % bright ansi colors
    \definecolor{darkgray}{gray}{0.25}
    \definecolor{lightred}{rgb}{1.0,0.39,0.28}
    \definecolor{lightgreen}{rgb}{0.48,0.99,0.0}
    \definecolor{lightblue}{rgb}{0.53,0.81,0.92}
    \definecolor{lightpurple}{rgb}{0.87,0.63,0.87}
    \definecolor{lightcyan}{rgb}{0.5,1.0,0.83}
    
    % commands and environments needed by pandoc snippets
    % extracted from the output of `pandoc -s`
    \DefineVerbatimEnvironment{Highlighting}{Verbatim}{commandchars=\\\{\}}
    % Add ',fontsize=\small' for more characters per line
    \newenvironment{Shaded}{}{}
    \newcommand{\KeywordTok}[1]{\textcolor[rgb]{0.00,0.44,0.13}{\textbf{{#1}}}}
    \newcommand{\DataTypeTok}[1]{\textcolor[rgb]{0.56,0.13,0.00}{{#1}}}
    \newcommand{\DecValTok}[1]{\textcolor[rgb]{0.25,0.63,0.44}{{#1}}}
    \newcommand{\BaseNTok}[1]{\textcolor[rgb]{0.25,0.63,0.44}{{#1}}}
    \newcommand{\FloatTok}[1]{\textcolor[rgb]{0.25,0.63,0.44}{{#1}}}
    \newcommand{\CharTok}[1]{\textcolor[rgb]{0.25,0.44,0.63}{{#1}}}
    \newcommand{\StringTok}[1]{\textcolor[rgb]{0.25,0.44,0.63}{{#1}}}
    \newcommand{\CommentTok}[1]{\textcolor[rgb]{0.38,0.63,0.69}{\textit{{#1}}}}
    \newcommand{\OtherTok}[1]{\textcolor[rgb]{0.00,0.44,0.13}{{#1}}}
    \newcommand{\AlertTok}[1]{\textcolor[rgb]{1.00,0.00,0.00}{\textbf{{#1}}}}
    \newcommand{\FunctionTok}[1]{\textcolor[rgb]{0.02,0.16,0.49}{{#1}}}
    \newcommand{\RegionMarkerTok}[1]{{#1}}
    \newcommand{\ErrorTok}[1]{\textcolor[rgb]{1.00,0.00,0.00}{\textbf{{#1}}}}
    \newcommand{\NormalTok}[1]{{#1}}
    
    % Define a nice break command that doesn't care if a line doesn't already
    % exist.
    \def\br{\hspace*{\fill} \\* }
    % Math Jax compatability definitions
    \def\gt{>}
    \def\lt{<}
    % Document parameters
    \title{Curds and Whey Regression Applied to
fMRIr}
\author{Chris Gagne and Alan Cowen}
    
    
    

    % Pygments definitions
    
\makeatletter
\def\PY@reset{\let\PY@it=\relax \let\PY@bf=\relax%
    \let\PY@ul=\relax \let\PY@tc=\relax%
    \let\PY@bc=\relax \let\PY@ff=\relax}
\def\PY@tok#1{\csname PY@tok@#1\endcsname}
\def\PY@toks#1+{\ifx\relax#1\empty\else%
    \PY@tok{#1}\expandafter\PY@toks\fi}
\def\PY@do#1{\PY@bc{\PY@tc{\PY@ul{%
    \PY@it{\PY@bf{\PY@ff{#1}}}}}}}
\def\PY#1#2{\PY@reset\PY@toks#1+\relax+\PY@do{#2}}

\expandafter\def\csname PY@tok@gd\endcsname{\def\PY@tc##1{\textcolor[rgb]{0.63,0.00,0.00}{##1}}}
\expandafter\def\csname PY@tok@gu\endcsname{\let\PY@bf=\textbf\def\PY@tc##1{\textcolor[rgb]{0.50,0.00,0.50}{##1}}}
\expandafter\def\csname PY@tok@gt\endcsname{\def\PY@tc##1{\textcolor[rgb]{0.00,0.27,0.87}{##1}}}
\expandafter\def\csname PY@tok@gs\endcsname{\let\PY@bf=\textbf}
\expandafter\def\csname PY@tok@gr\endcsname{\def\PY@tc##1{\textcolor[rgb]{1.00,0.00,0.00}{##1}}}
\expandafter\def\csname PY@tok@cm\endcsname{\let\PY@it=\textit\def\PY@tc##1{\textcolor[rgb]{0.25,0.50,0.50}{##1}}}
\expandafter\def\csname PY@tok@vg\endcsname{\def\PY@tc##1{\textcolor[rgb]{0.10,0.09,0.49}{##1}}}
\expandafter\def\csname PY@tok@m\endcsname{\def\PY@tc##1{\textcolor[rgb]{0.40,0.40,0.40}{##1}}}
\expandafter\def\csname PY@tok@mh\endcsname{\def\PY@tc##1{\textcolor[rgb]{0.40,0.40,0.40}{##1}}}
\expandafter\def\csname PY@tok@go\endcsname{\def\PY@tc##1{\textcolor[rgb]{0.53,0.53,0.53}{##1}}}
\expandafter\def\csname PY@tok@ge\endcsname{\let\PY@it=\textit}
\expandafter\def\csname PY@tok@vc\endcsname{\def\PY@tc##1{\textcolor[rgb]{0.10,0.09,0.49}{##1}}}
\expandafter\def\csname PY@tok@il\endcsname{\def\PY@tc##1{\textcolor[rgb]{0.40,0.40,0.40}{##1}}}
\expandafter\def\csname PY@tok@cs\endcsname{\let\PY@it=\textit\def\PY@tc##1{\textcolor[rgb]{0.25,0.50,0.50}{##1}}}
\expandafter\def\csname PY@tok@cp\endcsname{\def\PY@tc##1{\textcolor[rgb]{0.74,0.48,0.00}{##1}}}
\expandafter\def\csname PY@tok@gi\endcsname{\def\PY@tc##1{\textcolor[rgb]{0.00,0.63,0.00}{##1}}}
\expandafter\def\csname PY@tok@gh\endcsname{\let\PY@bf=\textbf\def\PY@tc##1{\textcolor[rgb]{0.00,0.00,0.50}{##1}}}
\expandafter\def\csname PY@tok@ni\endcsname{\let\PY@bf=\textbf\def\PY@tc##1{\textcolor[rgb]{0.60,0.60,0.60}{##1}}}
\expandafter\def\csname PY@tok@nl\endcsname{\def\PY@tc##1{\textcolor[rgb]{0.63,0.63,0.00}{##1}}}
\expandafter\def\csname PY@tok@nn\endcsname{\let\PY@bf=\textbf\def\PY@tc##1{\textcolor[rgb]{0.00,0.00,1.00}{##1}}}
\expandafter\def\csname PY@tok@no\endcsname{\def\PY@tc##1{\textcolor[rgb]{0.53,0.00,0.00}{##1}}}
\expandafter\def\csname PY@tok@na\endcsname{\def\PY@tc##1{\textcolor[rgb]{0.49,0.56,0.16}{##1}}}
\expandafter\def\csname PY@tok@nb\endcsname{\def\PY@tc##1{\textcolor[rgb]{0.00,0.50,0.00}{##1}}}
\expandafter\def\csname PY@tok@nc\endcsname{\let\PY@bf=\textbf\def\PY@tc##1{\textcolor[rgb]{0.00,0.00,1.00}{##1}}}
\expandafter\def\csname PY@tok@nd\endcsname{\def\PY@tc##1{\textcolor[rgb]{0.67,0.13,1.00}{##1}}}
\expandafter\def\csname PY@tok@ne\endcsname{\let\PY@bf=\textbf\def\PY@tc##1{\textcolor[rgb]{0.82,0.25,0.23}{##1}}}
\expandafter\def\csname PY@tok@nf\endcsname{\def\PY@tc##1{\textcolor[rgb]{0.00,0.00,1.00}{##1}}}
\expandafter\def\csname PY@tok@si\endcsname{\let\PY@bf=\textbf\def\PY@tc##1{\textcolor[rgb]{0.73,0.40,0.53}{##1}}}
\expandafter\def\csname PY@tok@s2\endcsname{\def\PY@tc##1{\textcolor[rgb]{0.73,0.13,0.13}{##1}}}
\expandafter\def\csname PY@tok@vi\endcsname{\def\PY@tc##1{\textcolor[rgb]{0.10,0.09,0.49}{##1}}}
\expandafter\def\csname PY@tok@nt\endcsname{\let\PY@bf=\textbf\def\PY@tc##1{\textcolor[rgb]{0.00,0.50,0.00}{##1}}}
\expandafter\def\csname PY@tok@nv\endcsname{\def\PY@tc##1{\textcolor[rgb]{0.10,0.09,0.49}{##1}}}
\expandafter\def\csname PY@tok@s1\endcsname{\def\PY@tc##1{\textcolor[rgb]{0.73,0.13,0.13}{##1}}}
\expandafter\def\csname PY@tok@kd\endcsname{\let\PY@bf=\textbf\def\PY@tc##1{\textcolor[rgb]{0.00,0.50,0.00}{##1}}}
\expandafter\def\csname PY@tok@sh\endcsname{\def\PY@tc##1{\textcolor[rgb]{0.73,0.13,0.13}{##1}}}
\expandafter\def\csname PY@tok@sc\endcsname{\def\PY@tc##1{\textcolor[rgb]{0.73,0.13,0.13}{##1}}}
\expandafter\def\csname PY@tok@sx\endcsname{\def\PY@tc##1{\textcolor[rgb]{0.00,0.50,0.00}{##1}}}
\expandafter\def\csname PY@tok@bp\endcsname{\def\PY@tc##1{\textcolor[rgb]{0.00,0.50,0.00}{##1}}}
\expandafter\def\csname PY@tok@c1\endcsname{\let\PY@it=\textit\def\PY@tc##1{\textcolor[rgb]{0.25,0.50,0.50}{##1}}}
\expandafter\def\csname PY@tok@kc\endcsname{\let\PY@bf=\textbf\def\PY@tc##1{\textcolor[rgb]{0.00,0.50,0.00}{##1}}}
\expandafter\def\csname PY@tok@c\endcsname{\let\PY@it=\textit\def\PY@tc##1{\textcolor[rgb]{0.25,0.50,0.50}{##1}}}
\expandafter\def\csname PY@tok@mf\endcsname{\def\PY@tc##1{\textcolor[rgb]{0.40,0.40,0.40}{##1}}}
\expandafter\def\csname PY@tok@err\endcsname{\def\PY@bc##1{\setlength{\fboxsep}{0pt}\fcolorbox[rgb]{1.00,0.00,0.00}{1,1,1}{\strut ##1}}}
\expandafter\def\csname PY@tok@mb\endcsname{\def\PY@tc##1{\textcolor[rgb]{0.40,0.40,0.40}{##1}}}
\expandafter\def\csname PY@tok@ss\endcsname{\def\PY@tc##1{\textcolor[rgb]{0.10,0.09,0.49}{##1}}}
\expandafter\def\csname PY@tok@sr\endcsname{\def\PY@tc##1{\textcolor[rgb]{0.73,0.40,0.53}{##1}}}
\expandafter\def\csname PY@tok@mo\endcsname{\def\PY@tc##1{\textcolor[rgb]{0.40,0.40,0.40}{##1}}}
\expandafter\def\csname PY@tok@kn\endcsname{\let\PY@bf=\textbf\def\PY@tc##1{\textcolor[rgb]{0.00,0.50,0.00}{##1}}}
\expandafter\def\csname PY@tok@mi\endcsname{\def\PY@tc##1{\textcolor[rgb]{0.40,0.40,0.40}{##1}}}
\expandafter\def\csname PY@tok@gp\endcsname{\let\PY@bf=\textbf\def\PY@tc##1{\textcolor[rgb]{0.00,0.00,0.50}{##1}}}
\expandafter\def\csname PY@tok@o\endcsname{\def\PY@tc##1{\textcolor[rgb]{0.40,0.40,0.40}{##1}}}
\expandafter\def\csname PY@tok@kr\endcsname{\let\PY@bf=\textbf\def\PY@tc##1{\textcolor[rgb]{0.00,0.50,0.00}{##1}}}
\expandafter\def\csname PY@tok@s\endcsname{\def\PY@tc##1{\textcolor[rgb]{0.73,0.13,0.13}{##1}}}
\expandafter\def\csname PY@tok@kp\endcsname{\def\PY@tc##1{\textcolor[rgb]{0.00,0.50,0.00}{##1}}}
\expandafter\def\csname PY@tok@w\endcsname{\def\PY@tc##1{\textcolor[rgb]{0.73,0.73,0.73}{##1}}}
\expandafter\def\csname PY@tok@kt\endcsname{\def\PY@tc##1{\textcolor[rgb]{0.69,0.00,0.25}{##1}}}
\expandafter\def\csname PY@tok@ow\endcsname{\let\PY@bf=\textbf\def\PY@tc##1{\textcolor[rgb]{0.67,0.13,1.00}{##1}}}
\expandafter\def\csname PY@tok@sb\endcsname{\def\PY@tc##1{\textcolor[rgb]{0.73,0.13,0.13}{##1}}}
\expandafter\def\csname PY@tok@k\endcsname{\let\PY@bf=\textbf\def\PY@tc##1{\textcolor[rgb]{0.00,0.50,0.00}{##1}}}
\expandafter\def\csname PY@tok@se\endcsname{\let\PY@bf=\textbf\def\PY@tc##1{\textcolor[rgb]{0.73,0.40,0.13}{##1}}}
\expandafter\def\csname PY@tok@sd\endcsname{\let\PY@it=\textit\def\PY@tc##1{\textcolor[rgb]{0.73,0.13,0.13}{##1}}}

\def\PYZbs{\char`\\}
\def\PYZus{\char`\_}
\def\PYZob{\char`\{}
\def\PYZcb{\char`\}}
\def\PYZca{\char`\^}
\def\PYZam{\char`\&}
\def\PYZlt{\char`\<}
\def\PYZgt{\char`\>}
\def\PYZsh{\char`\#}
\def\PYZpc{\char`\%}
\def\PYZdl{\char`\$}
\def\PYZhy{\char`\-}
\def\PYZsq{\char`\'}
\def\PYZdq{\char`\"}
\def\PYZti{\char`\~}
% for compatibility with earlier versions
\def\PYZat{@}
\def\PYZlb{[}
\def\PYZrb{]}
\makeatother


    % Exact colors from NB
    \definecolor{incolor}{rgb}{0.0, 0.0, 0.5}
    \definecolor{outcolor}{rgb}{0.545, 0.0, 0.0}



    
    % Prevent overflowing lines due to hard-to-break entities
    \sloppy 
    % Setup hyperref package
    \hypersetup{
      breaklinks=true,  % so long urls are correctly broken across lines
      colorlinks=true,
      urlcolor=blue,
      linkcolor=darkorange,
      citecolor=darkgreen,
      }
    % Slightly bigger margins than the latex defaults
    
    \geometry{verbose,tmargin=1in,bmargin=1in,lmargin=1in,rmargin=1in}
    
    
\begin{document}
    
    
    \maketitle
    
    

    



    

\section{Introduction}\label{introduction}

When making inferences on multiple responses (Y), performance on each
can be improved by pooling information across problems. Moreover, the
amount of improvement depends in part on how similar the problems are.
For fMRI data, inference is typically done for each individual voxel,
but we anticipate that gains can be made by combining inference across
multiple voxels. fMRI data is characterized by spatial correlations
across voxels, some of which is due to a functionally-modular
topographic organization of cortex. Therefore, this spatially correlated
signal may make fMRI particularly conducive to a technique that joins
inference across voxels. The particular techqnique we chose was the
``Curds and Whey Regression'' (CW) proposed by (Breiman and Friedman
1997)(BF).

Our project has three major parts. First, we wanted to replicate the
simulation results of Breiman and Friedman to verify that our
implementation was correct, and assess CW's performance across a range
of situations. Second, we wanted to apply CW to simulated fMRI data to
assess its performance in a more realistic simulation. Finally, we
applied CW to a real fMRI data set, in which we predict held out data in
order to choose scientific models of cortical organization.

\section{Background}\label{background}

\subsection{Setup}\label{setup}

Our setup is one of predicting multiple voxel response time courses
given multiple experimental predictors (stimuli presented to the
participant). We are interested in both predicting new voxel responses,
and estimating the `true' parameters underlying the voxel responses.

The voxel responses at a given timepoint is a row vector \$ \textbf{y'}
= (y\_1,\ldots{},y\_q) \$. We've assumed that each voxel's response can
be described as a linear combination $ \beta $ of responses to our
experimental manipulation \$ \textbf{x}=(x\_1,\ldots{},x\_p) \$ and
additional zero-mean error.

\[ y_i = \textbf{x} \beta_i + \epsilon_i \]
\[ y_q = \textbf{x} \beta_q + \epsilon_q \]

\subsubsection{OLS}\label{ols}

In the typical approach to fMRI analysis, we stack the y's and x's
across the n time points and solve using OLS, in which the estimation
for each y is solved independently of all others.

\[ \hat \beta_{ols} =  (\textbf{X}'\textbf{X})^{-1} \textbf{X}' \textbf{Y} \]
\[ \hat Y = \textbf{X} \hat \beta_{ols} \]

Although OLS is the minimum variance unbiased estimator, improvements in
prediction can be made by shrinking the estimates towards zero. Copas
(1983) showed that the expected slope of a regression of future y's on
our OLS estimates $ \hat y $ will be too large. Therefore, his method
and many others shrink the $ \hat y $ prior to prediction.

\subsubsection{Ridge}\label{ridge}

One such method is Ridge Regression which estimates the amount of
shrinkage $ \lambda $ using cross-validation. We currently use Ridge
on our data.

\[ \hat \beta_{rr} =  (\textbf{X}'\textbf{X} + I_p \lambda )^{-1} \textbf{X}' \textbf{Y} \]

\subsubsection{OLS-Curds-Whey}\label{ols-curds-whey}

Both Ridge Regression and Copas's method shrink the OLS estimates based
soley on x. BF proposed that further improvements can be made in
situations with multiple responses. Their method is based multplying the
OLS estimates by a shrinkage matrix S that incorporates the relationship
between multiple y's and x's.

\[  \hat Yhat = \textbf{X} \hat \beta_{ols} S^{cw} \]

BF derive the optimal shrinkage matrix (in an idealized setting), and
highlight its relation to a canonical correlations analysis (CCA) of x
and y. To see this, we start with the following combination of
covariances for y and x,

\[ Q = E(yy^T)^{-1}E(yx^T)E(xx^T)^{-1}E(xy^T) \]

which BF show relates to optimal shrinkage in the following way (where r
is the ratio of p/n):

\[ S^{cw} = ((1-r)I+rQ^{-T})^{-1} \]

In CCA, Q is usually decomposed into its eigenvectors and eigenvalues.

\[ Q = T C^2 T^{-1} \]

where T represents a new basis for Y in which the covariance between Y
and X is a diagonal matrix C, and diagonals are maximized across all
possible bases (for X and Y)(which are known the squared canonical
correlations). Doing this, we can re-write the optimal shrinkage matrix
as,

\[ S^{cw} = T D T^{-1} \] \[ D = \frac{c^2}{c^2 + r(1-c^2)} \]

where $ c^{2} $ are the diagonal elements of the \(C^2\).


Thus this decomposition uncovers what BF's shrinkage achieves.
Multiplying \(\hat Y\) by $ S^{cw} $ effectively
transforms y into a new basis where x and y are maximally correlated,
shrinks relative to the canonical correlations, and inversely transforms
back. Intuitively, CCA has found a new basis for our voxels y and
experimental manipulation x in which they have the strongest mutual
relationship. For example, an important dimension in Y space may be one
in which many voxels respond similarly to a certain experimental
manipulation. Shrinkage by D will preserve this dimension while
shrinking less important ones (ie. noise).

In practice, we estimate Q with our observed data:

\[ Q = E(Y^T Y)^{-1}E(Y^T X)E(X^T X)^{-1}E(X^T Y) \]

Then because our sample estimates for the values of $ c^{2} $ tend to
be too high, we adjust those as well. Using generalized cross
validation, we can approximate how much we should decrease the canonical
correlations.

\[ D = \frac{(1-r)(c^2-r)}{(1-r)^2c^2 + r^2(1-c^2)} \]

This is the D we use in our implementation. It is important to note that
when q\textgreater{}p, there are only p canonical correlations.
Therefore, like BF, we set those elements in D=0.

\subsubsection{Ridge-Curds-Whey}\label{ridge-curds-whey}

In situations where p is close to n, it may be beneficial to combine
ridge and CW shrinkage. To do this, you multiply the estimates found via
ridge regression by a similar shrinkage matrix.

\[  \hat Yhat = \textbf{X} \hat \beta_{rr} S^{cw} \]

Two differences to the above procedure are that the CCA was done on $
(Y, \hat Y\_\{ridge\}) $ rather than on (Y, X), and r is calculated
using 
\[ \frac{1}{N} trace ((\textbf{X}^T \textbf{X} + I_p \lambda )^{-1} \textbf{X}^T) \]

which can be considered the effective' degrees of freedom of the model.

\subsubsection{Double Ridge}\label{double-ridge}

As an alternative approach to using CW and CCA, we tried to estimate the
optimal shrinkage directly by predicting predicting held out Y from $
\hat Y $ using cross validation with added L2-regularization:

\[ \hat S = argmin_{S} (||Y –  S  \hat Y|| + \gamma|| S|| \]

This is essentially a second step of ridge regression (i.e.~ridge was
first used to predict Y from X on a voxel-by-voxel basis, then for each
voxel used to predict Y from \(\hat Y\) of voxels in a surrounding
neighborhood).

To determine the regularization parameter, \(\gamma\) (to be
distinguished \(\lambda\) from the first ridge regression of Y onto X),
we used 9-fold cross-validation in the estimation set. For each fold, \$
S\_k \$ was chosen as

\[ S_k = argmin_S(||Y_{estimation,training fold} – S \hat Y_{estimation, training fold}|| + \lambda||S|| \]

And used to produce

\[ \hat Y_{new, estimation, testing fold} \]

After cross-validation, we selected the $ \gamma $ that resulted in
the highest correlation between $ (Y , S \hat Y) $. This was done
separately for each voxel, i.

\section{Part I: Simulations (BF)}\label{part-i-simulations-bf}

\subsection{Summary}

The two primary goals of these simulations were to verify our
implementation of the CW shrinkage applied to OLS and Ridge Regression,
and to identify situations in which it is most beneficial. We ran 50
simulations on each of the 4 models (excluding Double Ridge) with each
of the following values of parameters:

\begin{itemize}
\itemsep1pt\parskip0pt\parsep0pt
\item
  n observations (200,400)
\item
  p parameters in our design matrix (11,21,51)
\item
  q responses y (5,10,20,50)
\item
  SNR (signal-to-noise ratio) (.01,.1,.5,1,2)
\item
  bsig: variance in the distribution of true B (.01,.1,.5,1,2)
\end{itemize}

\subsection{Data Generation}\label{data-generation}

The data for each simulation was generated in a similar fashion to BF.

\begin{enumerate}
\def\labelenumi{\arabic{enumi}.}
\item
  X: Each x was chosen as a random draw from multivariate guassian with
  a covariance $ \textbf{V} $ that was itself randomly generated for
  each simulation. Each entry in the covariance was random number from
  -1 to 1.
\item
  B: The true parameters were generated independently across x, but
  dependently across y. For each x, a random value was chosen from -1 to
  1 to be the mean of the true parameter B. The B were then sampled from
  that guassian with a width determined by our free parameter (bsig).
  With a small variance, the y's would have similar B, a situation that
  should favor CW.
\item
  F: The signal matrix was calculated as $ \beta XX^{T} \beta^{T} $.
  This was then used to adjust the error variance \(\sigma\) so that
  different values of bsig would not adjust the overall SNR. $
  \sigma $ was related to the overall signal so that SNR =
  mean(F)/sigma.
\item
  $ \epsilon $: Were independently drawn from a gaussian with mean
  zero and covariance $ I_q \sigma $ .
\item
  We then divided into training and test sets (centering within each),
  fit the model on the training set and predicted the test set. We
  assessed performance on both the training set (estimation of true B)
  and the test set (prediction accuracy).
\end{enumerate}

\subsection{Performance Metrics}\label{performance-metrics}

\begin{itemize}
\itemsep1pt\parskip0pt\parsep0pt
\item
  \textbf{Estimation error}: As fMRI analyses are often aimed at
  estimating the effects of some experimental manipulation, we wanted to
  assess how well the models estimated the true parameters in the
  training set. Following BF we used the following loss function, where
  V is the true covariance used to generate the X.
\end{itemize}

\[ est.mse = |\beta-\hat \beta|^T V |\beta - \hat \beta| \]

\begin{itemize}
\itemsep1pt\parskip0pt\parsep0pt
\item
  \textbf{Prediction accuracy}: For predictive performance, we measure
  the correlation of each response \$ y\_i\$ in the test set with its
  estimate \(\hat y_i\)
\end{itemize}

    \subsection{Results: Overall}\label{results-overall}

Looking at the average performance across all simulations, CW performed
better than OLS, ridge and CW performed similarly, and combining with
ridge with CW performed the best. However, depending on the situation
these relative performances shifted.



    \begin{center}
    \adjustimage{max size={0.9\linewidth}{0.9\paperheight}}{P215_Paper_files/P215_Paper_5_0.png}
    \end{center}
    { \hspace*{\fill} \\}
    
    \subsection{Results: Signal-to-Noise}\label{results-signal-to-noise}

Averaging across all other parameters, we looked how the signal-to-noise
ratio effected model performance. In general, prediction accuracy
increased with SNR, however estimation MSE did not. Importantly, the
effect of shrinking the estimates (either with ridge or CW) had
improvements over OLS in cases of low SNR (0.01, 0.1) for both
estimation and prediction.


    \begin{center}
    \adjustimage{max size={0.9\linewidth}{0.9\paperheight}}{P215_Paper_files/P215_Paper_7_0.png}
    \end{center}
    { \hspace*{\fill} \\}
    

    \begin{center}
    \adjustimage{max size={0.9\linewidth}{0.9\paperheight}}{P215_Paper_files/P215_Paper_8_0.png}
    \end{center}
    { \hspace*{\fill} \\}
    
    \subsection{Results: B variance}\label{results-b-variance}

Next, we averaged across all parameters except bsig, which determined
the variance of B for each predictor. Low variance meant that B's were
similar across voxels for a given predictor (left), whereas high
variance meant that each voxel could have different B's (right). At each
level of B variance, the SNR was kept constant by adjusting $ \sigma $
so the effects for this parameter on performance could be isolated.

The clearest benefit from CW is seen in prediction. For low variance in
B's, CW applied to both OLS and Ridge improves peformance (right
green/purple). This benefit decreases as the variance in the B's
increases. For estimation MSE, Ridge actually seems to do better than
CW, but more similuations need to be run, and the results might need to
be subset by other factors


    \begin{center}
    \adjustimage{max size={0.9\linewidth}{0.9\paperheight}}{P215_Paper_files/P215_Paper_10_0.png}
    \end{center}
    { \hspace*{\fill} \\}
    

    \begin{center}
    \adjustimage{max size={0.9\linewidth}{0.9\paperheight}}{P215_Paper_files/P215_Paper_11_0.png}
    \end{center}
    { \hspace*{\fill} \\}
    
    \section{Part II: Simulations (fMRI)}\label{part-ii-simulations-fmri}

\subsection{Summary}

In this section, we wanted to extend the simulations to data more
similar actual fMRI data. We simulated a simple experiment in which two
conditions occurred in a random sequence for a total of 400 seconds.
Each condition lasted 1 second (e.g.~like an image presentation) and
there was a delay between conditions. Several regions were simulated
which each responded to one condition or the other. These are shown in
the left two panels below. These spatial effects `B1' and `B2' were
multiplied by the design matrix to create a temporal response for each
voxel. Several types of noise were then added. For this particular
simulation, we added zero-mean guassian and AR1 noise to each voxel's
time course, and then spatial noise in the form of a gaussian random
field with 5 voxel FWHM. The spatial average of the final simulated data
is shown on the far right panel. The data simulation was done using a
combination of in-house software, and some basic functions
(e.g.~gaussian random field) from the R package neuRosim.



        
    \begin{center}
    \adjustimage{max size={0.9\linewidth}{0.9\paperheight}}{P215_Paper_files/P215_Paper_14_1.png}
    \end{center}
    { \hspace*{\fill} \\}
    
    \subsection{Search-Light Curds-Whey}\label{search-light-curds-whey}

Instead of applying CCA to all voxels and our two predictors, we
embedded CCA in a searchlight procedure. For each voxel, we took the 25
nearest voxels by euclidean distance. CCA was then calculated on these
voxels and the predictors to get a shrinkage matrix. This matrix was
applied to the predictions of voxel in the center of the searchlight,
and the procedure was repeated for all voxels. Below is an example of
the searchlight applied to the simulation.


    \begin{center}
    \adjustimage{max size={0.9\linewidth}{0.9\paperheight}}{P215_Paper_files/P215_Paper_16_0.png}
    \end{center}
    { \hspace*{\fill} \\}
    
    \subsection{Results}\label{results}

In this simulation, we only looked at OLS and OLS-CW. Assessed by the
same metrics as above, CW showed improvements over OLS.



    \begin{center}
    \adjustimage{max size={0.9\linewidth}{0.9\paperheight}}{P215_Paper_files/P215_Paper_19_0.png}
    \end{center}
    { \hspace*{\fill} \\}
    
    Looking at maps of the B estimates (below), it is apparent that OLS-CW
gave lower B weights to noise voxels that had B=0 (upper upper sections
of the brain maps).


    \begin{center}
    \adjustimage{max size={0.9\linewidth}{0.9\paperheight}}{P215_Paper_files/P215_Paper_21_1.png}
    \end{center}
    { \hspace*{\fill} \\}
    
    \section{Part III: fMRI Dataset}\label{part-iii-fmri-dataset}

\subsection{Summary}

Face perception is essential to human social interactions on a daily
basis. It enables us to identify a virtually unlimited number of people,
interpret a wide range of facial emotional cues, and draw rapid trait
inferences from facial appearance. These abilities are made possible by
a network of brain regions that seem to be largely dedicated to face
perception, including the fusiform face area (FFA), occipital face area
(OFA), and posterior superior temporal sulcus (pSTS).

While it is well understood that these brain regions activate in
response to face images, it is less clear what aspects of face
processing they each perform. Our approach is to measure brain activity
in response to a large set of naturalistic faces, then use
L2-regularized regression to see if we can predict brain activity as a
function of different features of the images. Figure 1 shows how this
works.

In general, we can compare the ability of different feature models to
predict voxels in different regions across cortex. Here, for simplicity,
we use only a semantic model. Membership of each face image in 102
semantic categories was independently determined by five naive raters
and then averaged. 53 terms described variant facial attributes
including 10 aspects of facial posture (e.g.~mouth open, squinting) and
43 emotional expressions. The other 49 terms described invariant facial
attributes such as gender and race.

The model was fit separately to the 1.92 hours of estimation data
collected within each individual voxel. To model the slow hemodynamic
response, variables within each model were assigned distinct
time-inseparable finite impulse response filters with four bins at
delays 2-4 s, 4-6 s, 6-8 s, and 8-10 s after stimulus onset. All model
parameters were simultaneously fit using L2-regularized linear
regression. The regularization parameter (λ) for regression was selected
with nine-fold cross-validation. Each voxel's prediction scores were
taken as the correlation coefficient (Pearson's r) between the actual
and predicted BOLD responses for that voxel. The optimal λ for all
voxels was determined by testing ten values and selecting the one for
which the maximum number of voxels had prediction scores exceeding a
threshold r value (0.06). The model-fitting procedures were performed
with in-house software written in Matlab (MathWorks). After fitting each
voxelwise encoding model, the proportion of variance explained in each
voxel was estimated by predicting the BOLD responses to the validation
set and correlating these predictions with the data.

\subsection{Methods}\label{methods}

Our goal was to use multi-output regression to improve the predictions
of the semantic model. To that end, we tested two different formulations
of multi-output regression as described above: searchlight
Ridge-Curds-and-Whey and searchlight Double-Ridge. For each voxel, we
only used the 25 nearest neighbors in either Ridge-CW or the second step
of Double-Ridge. For both methods, the out-of-sample response estimates
obtained during cross-validation to select λ in the estimation set were
taken as \(\hat Y\). For double-ridge, we used an additional 9-fold
cross-validation within the estimation set to choose \(\gamma\) and then
retrained the second ridge step across all of the estimation data to
estimate \(\hat S\) for this \(\gamma\). Finally, we adjusted our
predictions of Y in the validation set with \(S*\hat Y\) and correlated
these adjusted predictions with our measured BOLD responses (Y). These
correlations were compared to the correlations between our original
\(\hat Y\) and Y.


            \begin{Verbatim}[commandchars=\\\{\}]
{\color{outcolor}Out[{\color{outcolor}73}]:} <IPython.core.display.Image object>
\end{Verbatim}
        
    These figures show a scatter plot of the voxel-wise prediction
correlations when using searchlight Ridge-CW compared to Ridge (left),
and searchlight Double-Ridge compared to Ridge (right). Points lying
above the line correspond to voxels where there was an improvement using
the new multi-response methods. In both cases, multi-response methods
clearly improved correlations for the majority of relevant voxels,
albeit by a small margin.


    \begin{center}
    \adjustimage{max size={0.9\linewidth}{0.9\paperheight}}{P215_Paper_files/P215_Paper_25_0.png}
    \end{center}
    { \hspace*{\fill} \\}
    
    Below, we show the each voxel's prediction correlation on an inflated
representation of the subject's cortex for Ridge (left) vs Double-Ridge
(right). Outlined in green, is a region where noticeable improvement
occurred.


            \begin{Verbatim}[commandchars=\\\{\}]
{\color{outcolor}Out[{\color{outcolor}75}]:} <IPython.core.display.Image object>
\end{Verbatim}
        


    
    
    
\end{document}
